\chapter{Abstract}

Human pose measurement, estimation, and reconstruction technology has become a significant research field over the years. This technology quantifies human poses into data such as joint angles, segment lengths, orientations, and joint positions. These data have found applications in various fields, including medical diagnosis and rehabilitation, sports training guidance, and animation and game motion simulation, contributing to advancements in healthcare, athletic performance, and entertainment experiences.

Marker-based motion capture systems are favored by researchers for their accuracy but are limited by high costs, complex setups, and sensitivity to ambient light. In recent years, with the rapid development of machine learning, image recognition technology has emerged as a new trend in human pose measurement. While it eliminates the need for markers and specialized clothing, simplifying experiment preparation, it remains susceptible to occlusion. To address this, many researchers have proposed improved methods, such as incorporating IMU (Inertial Measurement Unit) information fusion to enhance the success rate of human pose reconstruction. However, existing methods often lack open-source code or rely on marker-based motion capture systems during reconstruction, limiting their applicability.

This study builds upon an open-source fusion method, establishing a series of processes, including camera calibration, create a personalized 3D human body model, time synchronization, and spatial calibration. These processes create workflows that allow individuals to input their measurement data for human pose reconstruction, moving beyond the limitations of existing datasets. The research confirms that integrating IMU information effectively improves the success rate of human pose reconstruction, achieving the goal of enabling human pose measurement and reconstruction in any environment without relying on marker-based motion capture systems. Additionally, this study explores the possibility of reducing the number of cameras used for data collection without compromising accuracy, aiming to enhance equipment mobility and simplify setup complexity.

\bigskip
\textbf{Key words:} human pose measurement, human pose reconstruction, image recognition, IMU, sensor fusion
% \end{abstractEN}