\begin{abstractEN}

The advancement of technology has significantly broadened the possibilities of research in fields 
such as biomechanics and physiological signals, surpassing the limitations of clinical experiments. 
Through computer simulation and analysis, researchers can also obtain complex results like neural 
signals, muscle force, and joint torque. 
This progress has had unprecedented implications in various domains, including clinical medicine, 
rehabilitation, and sports science. 
However, achieving accurate results in human motion simulation and analysis relies not only on 
the evaluation methods but also on the choice of models. 
While generic models simplify the model-building process, they fail to fully capture the unique 
characteristics of individual subjects. 
Thus, the development of subject-specific models is crucial, albeit challenging.

In this study, the combination of biomechanics software, OpenSim, and mathematical 
computing software, MATLAB, is used to estimate musculotendon parameters in musculoskeletal models 
through optimization methods.
The primary focus of the research centers around prediction tasks. 
Prior to parameter estimation, sensitivity analysis is performed to determine the desired tasks to 
be executed. 
Subsequently, multiple prediction tasks are executed to quantify the discrepancy between the 
predicted trajectories and the target trajectories, enabling the determination of parameter 
values for the evaluated muscles.
Finally, the optimal models resulting from the evaluation process are subjected to model validation 
to ensure their accuracy.
The methodology is demonstrated through several simulation cases using a widely used upper extremity 
musculoskeletal model, confirming the feasibility and effectiveness of the proposed methods.
Moreover, the study investigates the issue of parameter non-identifiability and affirms that 
engaging in multiple prediction tasks is an effective means to circumvent its influence.
In conclusion, the proposed methodology effectively estimates musculotendon parameters in 
musculoskeletal models, providing substantial support for future development of subject-specific models.

\textbf{Key words:} subject-specific musculoskeletal model, Hill-type muscle model, musculotendon parameter estimation, parameter non-identifiability, optimization
\end{abstractEN}