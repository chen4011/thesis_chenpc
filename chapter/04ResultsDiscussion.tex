\chapter{系統驗證之實驗結果分析與討論}
\fontsize{12pt}{18pt}\selectfont %字體大小,行距

% ------------------------- 4.0 ------------------------- %
% 概述
本章節會以第三章所介紹之方法進行延伸討論,以人體的上臂肌肉為主要研究對象,藉由執行不同的動作任務,
來評估上臂特定肌肉之參數,其中以 OpenSim 與 MATLAB 軟體作為模擬與分析的工具。

% 評估對象種類;動作任務分類
本研究之上臂肌肉以肱二頭肌群為主要評估對象,其中又分為長頭與短頭兩條肌肉,代表著該肌群需要兩個希爾式肌肉模型來模擬,
而每個模型欲評估參數有三個,分別為最大等長力量 ($F^\mathrm{M}_\mathrm{O}$)、
最佳肌纖維長度 ($L^\mathrm{M}_\mathrm{O}$) 與肌腱鬆弛長度($L^\mathrm{T}_\mathrm{S}$),
動作任務則主要以手肘彎曲為主,藉由不同的彎曲方向、肩膀位置與彎曲範圍作為分類。
另外模擬案例在套用第三章方法中,不管是敏感度分析、最佳化評估還是模型驗證,用於計算之運動軌跡皆是採用關節轉動速度,
其相比於轉動角度更能呈現出細微差異,在求解的肌肉參數精準度上,也將會有更好的效果。

% ------------------------- 4.1 ------------------------- %
\section{個人化三維人體模型建立實驗結果分析與討論}\label{ch4_skeleton_exp}
% 實驗設定;實驗執行;結果與討論
在章節~\ref{ch3_skeleton_method} 中已詳細描述了個人化三維人體模型建立的方法,
本節使用 TotalCapture Dataset~\cite{Trumble:BMVC:2017} 提供的影片資料及相機校正數據,
嘗試建立影片中受試者的個人化三維人體模型,並進行比較及討論。

\subsection{實驗設定}
% 把 total capture dataset s1 系列做完
本章節使用 TotalCapture Dataset 之 
s1\_acting1 \textasciitilde\ s1\_acting3、s1\_freestyle1 \textasciitilde\ s1\_freestyle3、s1\_rom1 \textasciitilde\ s1\_rom3 
共九組影片資料及相機校正資料進行實驗,
每組實驗皆取用 TotalCapture Dataset 提供之相機 1 與相機 8 影像資料、兩台相機之校正資訊,
及 .bvh 檔案中 HIERARCHY 部分提供之資訊(此資訊可由 Vicon 量測而得,因此以下將稱為 Vicon 三維人體模型)。
首先,使用影像資料進行 OpenPose 影像辨識,並利用 Pose2Sim 進行相機校正及三角測量計算,建立出個人化三維人體模型,
最後取前 60 幀的資訊,計算個人化三維人體模型之平均四肢長度與 Vicon 三維人體模型之四肢長度進行比較。

\subsection{誤差評估}
% 個人化三維人體模型建立結果與驗證
% 分別評估四肢的誤差,然後再綜合再一起評估,總結誤差大概在多少內
分別計算出自行建立個人化三維人體模型的四肢長度及 Vicon 三維人體模型長度後,將兩者相減得到誤差,並計算平均誤差,
結果如表~\ref{ch3_skeleton_compare} 所示,可以發現自行建立個人化三維人體模型之四肢長度與 Vicon 三維人體模型之四肢長度相當接近,
整體平均誤差為 25.7515 (mm)。
本研究推斷,
% TODO:推斷原因

\begin{table}[!ht]
   \caption[個人化三維人體模型建立結果與比較(mm)]{個人化三維人體模型建立結果與比較(mm)}
   \centering
   \label{ch3_skeleton_compare}
   \setlength{\tabcolsep}{3pt}
   \renewcommand\arraystretch{1.5}
   \resizebox{\textwidth}{!}{
    \begin{tabular}{c|S|S|S|S|S|S|S|S|S||S}
      & {s1\_acting1} & {s1\_acting2} & {s1\_acting3} & {s1\_freestyle1} & {s1\_freestyle2} & {s1\_freestyle3} & {s1\_rom1} & {s1\_rom2} & {s1\_rom3} & {average} \\
      \midrule[1.5pt]
      右大腿 & 15.5895 & 16.8009 & 5.8655 & 27.9576 & 24.0688 & 18.2016 & 9.7907 & 12.2067 & 1.5568 & 14.6709  \\
      右小腿 & 20.4312 & 0.7514 & 8.9552 & 11.4757 & 12.7413 & 2.5780 & 19.0733 & 20.5609 & 8.6443 & 11.6902  \\
      左大腿 & 20.1423 & 23.3496 & 14.8347 & 34.8388 & 25.4773 & 29.7342 & 18.8850 & 15.3576 & 24.7152 & 23.0372  \\
      左小腿 & 3.4795 & 7.8705 & 16.6157 & 15.0529 & 12.4442 & 15.5452 & 12.3024 & 3.2584 & 3.2571 & 9.9807  \\
      右上臂 & 37.1875 & 39.5958 & 43.2833 & 76.6637 & 59.6534 & 34.1348 & 59.4540 & 29.4289 & 32.0788 & 45.7200  \\
      右前臂 & 4.0619 & 20.9168 & 14.4801 & 59.1019 & 37.3311 & 10.6598 & 36.6949 & 4.3855 & 13.8587 & 22.3878  \\
      左上臂 & 47.6423 & 50.8824 & 51.3296 & 41.4413 & 48.9365 & 51.4196 & 36.5617 & 50.4505 & 53.2362 & 47.9889  \\
      左前臂 & 30.7583 & 24.3253 & 36.5486 & 19.8007 & 29.5842 & 34.7937 & 22.6062 & 35.9636 & 40.4441 & 30.5361  \\
      \midrule[1.5pt]
      average & 22.4116 & 23.0616 & 23.9891 & 35.7916 & 31.2796 & 24.6334 & 26.9210 & 21.4515 & 22.2239 & 25.7515  \\
   \end{tabular}}
\end{table}

\subsection{結論}
% 結論
由以上誤差評估可知,本方法的整體平均誤差約為 25.75 (mm),
若僅評估人體姿勢,不延伸應用於評估手指姿勢等細微動作,則此誤差並不會造成誤判的影響,因此證明本方法確實可行。

\clearpage

% ------------------------- 4.2 ------------------------- %
\section{單獨使用影像辨識估計姿態}
單獨 heatmap 的結果
\subsection{實驗設定}
% 實驗設定
123123
\subsection{實驗執行}
% 實驗執行
123123
\subsection{誤差評估}
% 誤差評估
123123
\subsection{結論}
% 結論
123123

% % ------------------------- 4.3 ------------------------- %
% \section{單獨做 IMU 的姿勢估計}
% 單獨 IMU 的結果
% \subsection{實驗設定}
% % 實驗設定
% 123123
% \subsection{實驗執行}
% % 實驗執行
% 123123
% \subsection{誤差評估}
% % 誤差評估
% 123123
% \subsection{結論}
% % 結論
% 123123

% ------------------------- 4.3 ------------------------- %
\section{使用影像辨識融合 IMU 於室內估計姿態}
sensor fusion 在室內的結果
\subsection{實驗設定}
% 實驗設定
123123
\subsection{實驗執行}
% 實驗執行
123123
\subsection{誤差評估}
% 誤差評估
123123
\subsection{結論}
% 結論
123123

% ------------------------- 4.4 ------------------------- %
\section{使用影像辨識融合 IMU 於室外估計姿態}
sensor fusion 在室外的結果
\subsection{實驗設定}
% 實驗設定
123123
\subsection{實驗執行}
% 實驗執行
123123
\subsection{誤差評估}
% 誤差評估
123123
\subsection{結論}
% 結論
123123

% ------------------------- 4.5 ------------------------- %
\section{小結}
% 回顧;下章節再討論;從驗證結果證實方法有效
本章節將上肢肌肉骨骼模型套用在所提出的研究方法,介紹了許多模擬案例的細節,如負重功能的新增、運動軌跡公式、
任務種類等資訊,除此之外也先透過該模型來展示敏感度分析的結果,提供後續的最佳化與模型驗證的任務挑選,
最主要的目的是要完成上肢特定肌肉之參數評估。下個章節將依據第三章的研究方法、第四章的前置作業,
完整介紹參數評估案例,並針對評估結果進行探討。

\clearpage