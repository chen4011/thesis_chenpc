\chapter{系統驗證之實驗結果分析與討論}
\fontsize{12pt}{18pt}\selectfont %字體大小,行距

% ------------------------- 4.0 ------------------------- %
% 概述
本章節會以第三章所介紹之方法進行延伸討論,以人體的上臂肌肉為主要研究對象,藉由執行不同的動作任務,
來評估上臂特定肌肉之參數,其中以 OpenSim 與 MATLAB 軟體作為模擬與分析的工具。

% 評估對象種類;動作任務分類
本研究之上臂肌肉以肱二頭肌群為主要評估對象,其中又分為長頭與短頭兩條肌肉,代表著該肌群需要兩個希爾式肌肉模型來模擬,
而每個模型欲評估參數有三個,分別為最大等長力量 ($F^\mathrm{M}_\mathrm{O}$)、
最佳肌纖維長度 ($L^\mathrm{M}_\mathrm{O}$) 與肌腱鬆弛長度($L^\mathrm{T}_\mathrm{S}$),
動作任務則主要以手肘彎曲為主,藉由不同的彎曲方向、肩膀位置與彎曲範圍作為分類。
另外模擬案例在套用第三章方法中,不管是敏感度分析、最佳化評估還是模型驗證,用於計算之運動軌跡皆是採用關節轉動速度,
其相比於轉動角度更能呈現出細微差異,在求解的肌肉參數精準度上,也將會有更好的效果。

% ------------------------- 4.1 ------------------------- %
\section{人體立體骨架建立實驗結果分析與討論}\label{ch4_skeleton_exp}
% 實驗設定;實驗執行;結果與討論
在章節~\ref{ch3_skeleton_method} 中已詳細描述了人體立體骨架建立的方法,
本節使用 TotalCapture Dataset~\ref{Trumble:BMVC:2017} 提供的影片資料及相機校正數據,
嘗試建立影片中受試者的立體骨架,並進行比較及討論。

\subsection{實驗設定}
% 把 total capture dataset s1 系列做完
本章節使用 TotalCapture Dataset 之 s1\_acting1 ~ s1\_acting3、s1\_freestyle1 ~ s1\_freestyle3、s1\_rom1 ~ s1\_rom3 共九組影片資料及相機校正資料進行實驗,
每組實驗皆取用 TotalCapture Dataset 提供之相機 1 與相機 8 影像資料、兩台相機之校正資訊,
及 .bvh 檔案中 HIERARCHY 部分提供之資訊(此資訊可由 Vicon 量測而得,因此以下將稱為 Vicon 立體骨架)。
首先,使用影像資料進行 OpenPose 影像辨識,並利用 Pose2Sim 進行相機校正及三角測量計算,建立出立體立體骨架,
最後取前 200 幀的計算立體骨架之平均四肢長度與 Vicon 立體骨架之四肢長度進行比較。

\subsection{誤差評估}
% 立體骨架建立結果與驗證
% 分別評估四肢的誤差,然後再綜合再一起評估,總結誤差大概在多少內
分別計算出自行建立立體骨架的四肢長度及 Vicon 立體骨架長度後,將兩者相減得到誤差,並計算平均誤差,
結果如表~\ref{ch3_skeleton_compare} 所示,可以發現自行建立立體骨架之四肢長度與 Vicon 立體骨架之四肢長度相當接近,
下半身的平均誤差較小,約 10.6125 (mm),上半身的平均誤差較大,約 24.93 (mm),整體平均誤差為 17.77 (mm)。
本研究推斷,
% TODO:由於擷取的前 200 幀受試者執行的動作多為...減少擷取的幀數,縮減到只剩 T pose 看看結果。

\begin{table}[!ht]
   \caption[立體骨架建立結果與比較(mm)]{立體骨架建立結果與比較(mm)}
   \centering
   \label{ch3_skeleton_compare}
   \setlength{\tabcolsep}{3pt}
   \renewcommand\arraystretch{1.5}
   \resizebox{\textwidth}{!}{
    \begin{tabular}{c|S|S|S|S|S|S|S|S|S|S}
      & {s1\_acting1} & {s1\_acting2} & {s1\_acting3} & {s1\_freestyle1} & {s1\_freestyle2} & {s1\_freestyle3} & {s1\_rom1} & {s1\_rom2} & {s1\_rom3} & {average} \\
      \midrule[2pt]
      右大腿 & 11.8657 & 10.7402 & 4.6752 & 13.0712 & 7.1899 & 8.4646 & 3.1011 & 0.7828 & 0.8141 & 6.7450 \\
      右小腿 & 1.7968 & 8.8412 & 17.8035 & 9.7986 & 12.1897 & 27.4657 & 3.5082 & 10.3631 & 9.3986 & 11.2406 \\
      左大腿 & 7.9281 & 8.3736 & 7.6390 & 5.5537 & 5.9827 & 17.0991 & 18.1283 & 17.5229 & 19.9397 & 12.0186 \\
      左小腿 & 7.6483 & 19.7528 & 7.2523 & 4.5994 & 10.8489 & 22.3818 & 7.7568 & 21.2137 & 10.5594 & 12.4459 \\
      右上臂 & 14.7551 & 32.1174 & 29.2950 & 29.3206 & 28.0889 & 38.5386 & 34.3777 & 30.0131 & 26.8481 & 29.2616 \\
      右前臂 & 32.1942 & 17.1299 & 20.2298 & 18.8817 & 21.8080 & 30.8246 & 17.6279 & 16.6320 & 12.6568 & 20.8872 \\
      左上臂 & 25.2333 & 24.0038 & 15.8676 & 22.8632 & 19.9119 & 46.8964 & 33.9388 & 34.0298 & 31.9454 & 28.2989 \\
      左前臂 & 21.5023 & 27.5297 & 7.0156 & 19.7651 & 16.2160 & 16.3309 & 26.4582 & 29.4440 & 27.2349 & 21.2774 \\
      \midrule
      average & 15.3655 & 18.5611 & 13.7222 & 15.4817 & 15.2795 & 26.0002 & 18.1121 & 20.0002 & 17.4246 & 17.7719 \\
   \end{tabular}}
\end{table}

\subsection{結論}
% 結論
由以上誤差評估可知,本方法的整體平均誤差約為 17.77 (mm),
若僅評估人體姿勢,不延伸應用於評估手指姿勢等細微動作,則此誤差並不會造成誤判的影響,因此證明本方法確實可行。

\clearpage

% ------------------------- 4.2 ------------------------- %
\section{單獨作影像辨識的姿勢估計}
單獨 heatmap 的結果
\subsection{實驗設定}
% 實驗設定
123123
\subsection{實驗執行}
% 實驗執行
123123
\subsection{誤差評估}
% 誤差評估
123123
\subsection{結論}
% 結論
123123

% ------------------------- 4.3 ------------------------- %
\section{單獨做 IMU 的姿勢估計}
單獨 IMU 的結果
\subsection{實驗設定}
% 實驗設定
123123
\subsection{實驗執行}
% 實驗執行
123123
\subsection{誤差評估}
% 誤差評估
123123
\subsection{結論}
% 結論
123123

% ------------------------- 4.4 ------------------------- %
\section{sensor fusion 室內實驗}
sensor fusion 在室內的結果
\subsection{實驗設定}
% 實驗設定
123123
\subsection{實驗執行}
% 實驗執行
123123
\subsection{誤差評估}
% 誤差評估
123123
\subsection{結論}
% 結論
123123

% ------------------------- 4.5 ------------------------- %
\section{sensor fusion 室外實驗}
sensor fusion 在室外的結果
\subsection{實驗設定}
% 實驗設定
123123
\subsection{實驗執行}
% 實驗執行
123123
\subsection{誤差評估}
% 誤差評估
123123
\subsection{結論}
% 結論
123123

% ------------------------- 4.6 ------------------------- %
\section{小結}
% 回顧;下章節再討論;從驗證結果證實方法有效
本章節將上肢肌肉骨骼模型套用在所提出的研究方法,介紹了許多模擬案例的細節,如負重功能的新增、運動軌跡公式、
任務種類等資訊,除此之外也先透過該模型來展示敏感度分析的結果,提供後續的最佳化與模型驗證的任務挑選,
最主要的目的是要完成上肢特定肌肉之參數評估。下個章節將依據第三章的研究方法、第四章的前置作業,
完整介紹參數評估案例,並針對評估結果進行探討。

\clearpage