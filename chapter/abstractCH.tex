\chapter{中文摘要}
\fontsize{12pt}{18pt}\selectfont

人體姿態量測、估計與重建技術經過多年的發展,已成為一個重要的研究領域。人體姿態量測與重建技術可將人體姿勢量化為關節角度、肢段長度、朝向及關節位置等數據,並應用於醫療診斷與復健、運動訓練指導,以及動畫及遊戲動作模擬等領域,有助於提升醫療水平、運動表現和娛樂體驗。

% 光標記動作捕捉系統因其量測精準的特性,廣受學者們喜愛,但其高昂的成本、複雜的設備架設及對環境光的要求,限制了其應用範圍。近年來,隨著機器學習技術的快速發展,無需光標記及特定實驗服裝的影像辨識技術逐漸成為人體姿態量測的新趨勢,大幅簡化實驗準備工作,但容易受到物體遮擋的影響。為了解決這個問題,許多學者提出了使用 IMU 資訊融合的改進方法,以提高人體姿態重建的成功率。然而,現有的方法多未公開程式碼,或在重建過程中仍需參考光標記動作捕捉系統的資訊,限制了其應用範圍。
光標記動作捕捉系統因其量測精準的特性,廣受學者們喜愛,但其高昂的成本、複雜的設備架設及對環境光的要求,限制其應用範圍。近年來,隨著機器學習技術的快速發展,無需光標記及特定實驗服裝的影像辨識技術逐漸成為人體姿態量測的新趨勢,大幅簡化實驗準備工作,但容易受到物體遮擋的影響。為解決這個問題,許多學者提出使用 IMU 資訊融合的改進方法,以提高人體姿態重建的成功率。然而,現有的方法仍需參考光標記動作捕捉系統的資訊,限制其應用範圍。

% 本研究基於一個開源的融合方法進行改進,建立一系列流程,包含相機校正、建立個人化三維人體模型、時間同步及空間校正,讓每位受試者皆可自行輸入量測資料,進行人體姿態重建,不再侷限於學者們發表的資料集。本研究證實融合 IMU 資訊可以有效提高人體姿態重建的成功率,並達成不需光標記動作捕捉系統輔助,在任何環境下皆可進行人體姿態量測及重建的目標。此外,本研究進一步探討在不影響準確度的前提下,減少使用相機數量進行數據蒐集的可能性,以提高設備的機動性及降低器材架設的複雜度。
本研究基於一個開源的融合方法進行改進,建立一系列流程,包含相機校正、建立個人化三維人體模型、時間同步及空間校正,讓每位受試者皆可自行輸入量測資料,進行人體姿態重建。本研究證實融合 IMU 資訊可以有效使人體姿態重建的成功率提升約 23.8\%,且對於誤差約有 5 \textasciitilde\ 10 (mm) 的改善,並達成不需光標記動作捕捉系統輔助,可在任何環境下進行人體姿態量測及重建的目標。此外,本研究進一步探討在不影響準確度的前提下,減少使用相機數量進行數據蒐集的可能性,以提高設備的機動性及降低器材架設的複雜度。

\bigskip
\textbf{關鍵字:} 人體姿態量測、人體姿態重建、影像辨識、IMU、感測器融合

% \end{abstractCH}