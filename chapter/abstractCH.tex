\begin{abstractCH}
\fontsize{12pt}{18pt}\selectfont

科技日益發展,電腦模擬與分析使得研究學者在生物力學、生理訊號等研究不再局限於臨床實驗的探討,
透過軟體模擬亦可得到複雜的分析結果,像是神經訊號、肌肉力量、關節扭矩等資訊,
不管是在臨床醫學、醫療復健還是運動科學等領域,皆帶來了前所未有的影響。在人體動作模擬與分析中,
除了模擬過程的計算評估方法會影響準確度外,模型的選擇亦影響結果重大,使用通用模型雖能減去模型建立的繁雜步驟,
但模擬結果並不能完全代表受試者本人,因此在個人化的模型建立上是必要的,不過同時也充滿了挑戰性。

本研究結合生物力學軟體 OpenSim 與數學計算軟體 MATLAB,以最佳化方法來估計肌肉骨骼模型中的肌肉肌腱參數,
整體研究以運動軌跡預測任務作為核心,參數估計前利用敏感度分析結果來選擇欲執行任務,
再透過多預測任務的執行,以預測軌跡與目標軌跡間的誤差來尋找欲評估肌肉之參數值,
而評估完成得到的最佳模型則需經過模型驗證的考驗。一連串的研究方法流程以數個模擬案例來呈現,
藉由普及的上肢肌肉骨骼模型來驗證方法的可行性與有效性,此外亦探討關於參數不可識別性問題,
並證實多預測任務可有效地避免其影響。綜合上述,所提出之研究方法能有效評估肌肉骨骼模型中的肌肉肌腱參數,
對於未來在個人化模型的建立上,將具有實質上的幫助。

\bigskip
\textbf{關鍵字:} 個人化肌肉骨骼模型、希爾式肌肉模型、肌肉肌腱參數評估、參數不可識別性、最佳化

\end{abstractCH}