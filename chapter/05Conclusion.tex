\chapter{結論與未來工作}
\fontsize{12pt}{18pt}\selectfont

% ------------------------- 6.0 ------------------------- %
在本論文中,首先於第一章介紹人體姿態量測、估計與重建的研究背景,以及每種方法的量測限制,衍生出本論文的動機與目的;接著於第二章介紹相關的研究文獻,並針對相關的文獻及應用進行回顧與探討;再來於第三章本研究所使用的研究方法,透過整系列的研究方法來進行實驗設置及資料前處理,旨在減少設備成本及架設時間,並使每位受試者都可以自行處理蒐集到的資料並進行感測器融合,從而重建人體姿態;第四章驗證減少設備使用數量的可行性、確認建立個人化三維人體模型的準確度,並藉由選擇有無融合 IMU 資訊驗證影像辨識融合 IMU 資訊可以有效提高人體姿態重建的成功率,並且解除對 Vicon 的依賴。最後,本章節將依序列出本研究的成果與貢獻,並提出三項未來工作,作為本研究可延伸之方向。

% ------------------------- 6.1 ------------------------- %
\section{研究成果與貢獻}
下方將條列出數個關於本研究之成果與貢獻,讓讀者除回顧本論文之核心方法外,也能清楚瞭解該論文之成果與貢獻處。

\begin{itemize}
    \item \textbf{探討減少相機使用數量對於人體姿態估計精準度的影響}
    \\ 本研究基於文獻~\cite{Zhang_2020_CVPR},使用其提出的感測器融合方法及 TotalCapture Dataset ~\cite{Trumble:BMVC:2017} 進行探討,TotalCapture Dataset 提供 8 台相機的影像資料、13 個 IMU 資料及 Vicon 資料的。本研究固定輸入 8 個 IMU 資料,並藉由選擇輸入相機數量及相機配對,探討減少相機使用數量對於人體姿態估計精準度的影響。
    \clearpage
    \item \textbf{應用影像辨識及三角測量計算建立三維人體模型}
    \\ 本研究提出使用影像辨識及三角測量計算方法建立三維人體模型,取代由 Vicon 資訊建立的三維人體模型,並透過該模型將 IMU 朝向資料轉換成三維位置資料,以供後續進行感測器融合,並進行人體姿態重建。
    \item \textbf{建立用於輸入感測器融合程式的資料前處理流程}
    \\ 本研究提出一系列資料前處理流程,包含相機校正、個人化三維人體模型建立、時間同步、空間校正等流程,以供每位受試者皆可自行處理資料,並輸入感測器融合程式進行人體姿態重建。
    \item \textbf{驗證影像辨識融合 IMU 資訊可以有效提高人體姿態重建的成功率}
    \\ 本研究進行蹲站、開合跳、折返跑、熱身運動等實驗,並透過選擇有無融合 IMU 資訊重建人體姿態,來驗證影像辨識融合 IMU 資訊可以有效提高人體姿態重建的成功率,並且在不需 Vicon 資訊的情況下,也可進行人體姿態重建。
\end{itemize}

% ------------------------- 6.2 ------------------------- %
\section{未來工作}
下方將列舉幾個未來工作,可作為本研究之延伸方向:

\begin{itemize}
    \item \textbf{解除對背景、衣著的限制}
    \\ 本研究目前使用的影像辨識模型由於訓練資料庫較單一,對於受試者、服裝、環境的容忍度較低,難以廣泛應用,因此未來可嘗試解除對背景、衣著的限制,使得該方法更具廣泛性。
    \item \textbf{探討相機擺放角度對人體姿態重建的影響}
    \\ 造成影像辨識融合 IMU 資訊估計人體姿態的誤差的原因除影像辨識模型的精準度及 IMU 資訊的精準度外,相機擺放角度也是一個重要因素,因此未來可探討相機擺放角度對人體姿態重建的影響。
    \item \textbf{量化重建姿態與原始姿態的相似程度}
    \\ 目前本研究僅透過視覺觀察重建姿態與原始姿態的相似程度,未來可透過量化的方法來評估重建姿態與原始姿態的相似程度,以提高評估的客觀性。
\end{itemize}

\clearpage