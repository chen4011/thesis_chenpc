\chapter{緒論}
\fontsize{12pt}{18pt}\selectfont

% ------------------------- 1.1 ------------------------- %
\section{前言}
% 人體動作模擬發展;應用
隨著感測器、電腦視覺及機器學習的發展,人體姿態量測估計與重建技術逐漸成為一個重要的研究領域。透過這項技術,可以將人體姿勢量化,例如計算關節角度、肢段長度、朝向及關節位置等,並應用於各個領域。在醫學領域,這些數據可以協助醫師進行診斷,並幫助治療師進行復健治療~\cite{tsakanikas2020evaluating}~\cite{zhao2017imu},從而提高醫療水平,讓病人更好地康復。在運動領域,這些數據可以幫助運動訓練師更好地指導運動員,提高訓練效率。在娛樂領域,這些數據可以幫助動畫師及遊戲開發者更好地估計人體動作,使得動畫及遊戲更加逼真,進一步創造娛樂種類的多樣性。

% 人體動作模擬種類
現今應用於人體姿態量測的感測器有許多種類,例如光學動作捕捉系統、各式相機、慣性感測器等。光學動作捕捉系統是目前最常被使用的感測器,其量測精準,動作重建方便,成為許多影視公司及動作分析實驗室的必備設備。深度相機則是一種使用深度影像辨識技術進行人體姿態量測的感測器,其不需貼標記點,且受試者無需穿著特殊服裝,因此在實驗設備的準備上較為簡單。慣性感測器則是一種使用 IMU (Inertial Measurement Unit) 進行人體姿態量測的感測器,其不受環境光影響,且較不受遮擋,因此在實驗環境的設定上較為自由。
另有外骨骼動作捕捉系統、壓力感測器等,這些感測器在特定領域有其獨特的優勢,例如外骨骼動作捕捉系統可以輔助行動不便的人進行復健治療,壓力感測器在體育運動中的應用等。

% ------------------------- 1.2 ------------------------- %
\section{研究動機與目的}
% 困境、動機、目的
通過人體姿態的量測與估計及重建,可以將人體的姿勢與動作數據化,目前最常被用於量測人體姿態及重建的系統為光標記動作捕捉系統,例如 OptiTrack、Vicon 等系統,其量測精準,動作重建方便,成為許多影視公司及動作分析實驗室的必備設備,但因為光標記系統的量測方法以紅外光為通訊媒介,因此量測環境需要嚴格控管環境光,且架設一個實驗場地需耗費許多時間,設備的成本也相對高昂,因此提高了人體姿態量測的使用門檻。

由於光標記動作捕捉系統的門檻限制及現今機器學習快速發Z展,使用影像辨識重建人體動作的方法越來越受到重視,藉由電腦視覺及機器學習辨別出影像中人體關節點的位置,從而進行人體姿態估計及重建,其事前準備工作簡單,受試者無需黏貼標記點,但是影像辨識容易受到自體或是周圍障礙物的遮擋,導致無法辨識出受遮蔽的關節點。為了解決這個問題,許多學者提出了改進方法,有些學者提出基於機器學習的方法,有些學者則提出使用 IMU 資訊融合,藉由 IMU 較不受遮擋的特性,改善影像辨識在遮擋情況下無法重建的問題。目前雖有許多學者提出有效且準確的融合方法,例如  \textit{3D Poses in the Wild} (3DPW) ~\cite{vonMarcard2018},但這些方法大多未公開程式碼,或是有些方法在重建人體姿態的過程中有參考 Vicon 資訊。因此,在時間有限的情況下,本研究將基於一個程式開源的融合方法進行改進,旨在證實融合 IMU 資訊可以有效提高人體姿態重建的成功率,並達成不需 Vicon 輔助,在任何環境下皆可進行人體姿態量測及重建的目標,讓這項技術不再侷限於實驗室,而是走入人類的日常生活。

本研究所使用的融合方法,在三維人體模型的建立及定位人體在空間中的位置時,皆參考了 Vicon 的資訊,且使用的影像辨識模型因為訓練資料庫較單一,對於受試者、服裝、環境的容忍度較低,難以廣泛應用。因此,本研究嘗試更改三維人體模型的建立方法及影像辨識模型,希望在不需 Vicon 資訊,也不需重新訓練影像辨識模型的情況下,可直接輸入量測資料,重建人體姿態,以期增加影像辨識及 IMU 融合方法的應用性。此外,本研究進一步探討在不影響準確度的前提下,減少使用相機數量進行數據蒐集的可能性,以期提高設備的機動性及降低器材架設的複雜度。

% 寫目前的研究成果和結論,就是誤差只增加0.5 cm,但是重建的成功率提高8~10%左右→本研究

% ------------------------- 1.3 ------------------------- %
\section{論文架構}
本論文共含有五個章節,其架構如下:

\begin{itemize}
    \item \textbf{第一章:緒論}
    \\ 介紹本論文之研究背景,由研究背景的需求與困境當中,衍生出本研究之動機與目的,闡述本論文之核心目標。
    \item \textbf{第二章:文獻回顧}
    \\ 針對該領域現存的研究進行介紹與整理,包括動作捕捉系統、人體模型種類、時間同步方法、感測器融合演算法等文獻,進行回顧與討論。
    \item \textbf{第三章:研究方法}
    \\ 介紹本研究所改善之人體姿態估計方法,包含實驗系統設定、相機校正、三維人體模型建立、時間同步及空間校正等方法,
    除此之外也進一步探討減少相機使用數量對於人體姿態估計精準度的影響。
    \item \textbf{第四章:系統驗證之實驗結果與討論}
    \\ 針對第三章所使用之方法,探討減少用於人體姿態估計的相機數量的可能性,評估三維人體模型建立的準確性及討論使用 IMU 進行感測器融和重建人體姿態的必要性。
    \item \textbf{第五章:結論與未來工作}
    \\ 總結本研究之成果與貢獻,並給予適當的建議,作為未來本研究之延伸方向。
\end{itemize}

\clearpage