\chapter{緒論}
\fontsize{12pt}{18pt}\selectfont

% ------------------------- 1.1 ------------------------- %
\section{前言}
% 人體動作模擬發展;應用

% 人體動作模擬種類

% 模型選擇;通用模型;個人化模型

% ------------------------- 1.2 ------------------------- %
\section{研究動機與目的}
% 困境、動機、目的
通過人體姿態的量測與估計及重建,可以將人體的姿勢與動作數據化,從而改善運動員的姿勢、協助影視動畫及遊戲人物的動作模擬、幫助醫學研究及醫療施行~\cite{tsakanikas2020evaluating}~\cite{zhao2017imu},目前最常被用於量測人體姿態及重建的系統為光標記動作捕捉系統,例如 OptiTrack、Vicon 等系統,其量測精準,動作重建方便,成為許多影視公司及動作分析實驗室的必備設備,但因為光標記系統的量測方法以紅外光為通訊媒介,因此量測環境需要嚴格控管環境光,且架設一個實驗場地需耗費許多時間,設備的成本也相對高昂,因此提高了人體姿態量測的使用門檻。

由於光標記動作捕捉系統的門檻限制及現今機器學習快速發展,使用影像辨識重建人體動作的方法越來越受到重視,藉由電腦視覺及機器學習辨別出影像中人體關節點的位置,從而進行人體姿態估計及重建,其事前準備工作簡單,受試者無需黏貼標記點,但是影像辨識容易受到自體或是周圍障礙物的遮擋,導致無法辨識出受遮蔽的關節點。為了解決這個問題,許多學者提出了改進方法,有些學者提出基於機器學習的方法,有些學者則提出使用 IMU 資訊融合,藉由 IMU 較不受遮擋的特性,改善影像辨識在遮擋情況下無法重建的問題。目前雖有許多學者提出有效且準確的融合方法,例如 3DPW~\cite{vonMarcard2018},但這些方法大多未公開程式碼,或是有些方法在重建人體姿態的過程中有參考 Vicon 資訊。因此,在時間有限的情況下,本研究將基於一個程式開源的融合方法進行改進,旨在證實融合 IMU 資訊可以有效提高人體姿態重建的成功率,並達成不需 Vicon 輔助,在任何環境下皆可進行人體姿態量測及重建的目標。

本研究所使用的融合方法,在三維人體模型的建立及定位人體在空間中的位置時,皆參考了Vicon的資訊,且使用的影像辨識模型因為訓練資料庫較單一,對於受試者、服裝、環境的容忍度較低,難以廣泛應用。因此,本研究嘗試更改三維人體模型的建立方法及影像辨識模型,希望在不需Vicon資訊,也不需重新訓練影像辨識模型的情況下,可直接輸入量測資料,重建人體姿態,以期增加影像辨識及IMU融合方法的應用性。此外,本研究進一步探討在不影響準確度的前提下,減少使用相機數量進行數據蒐集的可能性,以期提高設備的機動性及降低器材架設的複雜度。

% 寫目前的研究成果和結論,就是誤差只增加0.5 cm,但是重建的成功率提高8~10%左右→本研究

% ------------------------- 1.3 ------------------------- %
\section{論文架構}
本論文共含有五個章節,其架構如下:

\begin{itemize}
    \item \textbf{第一章:緒論}
    \\ 介紹本論文之研究背景,由研究背景的需求與困境當中,衍生出本研究之動機與目的,闡述本論文之核心目標。
    \item \textbf{第二章:文獻回顧}
    \\ 針對該領域現存的研究進行介紹與整理,包括動作捕捉系統、人體模型種類、時間對齊方法、感測器融合演算法等文獻,進行回顧與討論。
    \item \textbf{第三章:研究方法}
    \\ 介紹本研究所改善之人體姿態估計方法,包含實驗系統設定、相機校正、三維人體模型建立、時間空間對齊等方法,
    除此之外也進一步探討減少相機數量對進行人體姿態估計的影響。
    \item \textbf{第四章:系統驗證之實驗結果與討論}
    \\ 針對第三章所使用之方法,探討減少用於人體姿態估計的相機數量的可能性,評估三維人體模型建立的準確性及討論使用 IMU 進行感測器融和重建人體姿態的必要性。
    \item \textbf{第五章:結論與未來工作}
    \\ 總結本研究之成果與貢獻,並給予適當的建議,作為未來本研究之延伸方向。
\end{itemize}

\clearpage