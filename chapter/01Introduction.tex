\chapter{緒論}
\fontsize{12pt}{18pt}\selectfont

% ------------------------- 1.1 ------------------------- %
\section{前言}
% 人體動作模擬發展;應用

% 人體動作模擬種類

% 模型選擇;通用模型;個人化模型

% ------------------------- 1.2 ------------------------- %
\section{研究動機與目的}
% Vicon 骨架取代
% 時間軸對齊方式

% 困境;動機;目的

% ------------------------- 1.3 ------------------------- %
\section{論文架構}
本論文共含有六個章節,其架構如下:

\begin{itemize}
    \item \textbf{第一章:緒論}
    \\ 介紹本論文之研究背景,由研究背景的需求與困境當中,衍生出本研究之動機與目的,闡述本論文之核心目標。
    \item \textbf{第二章:文獻回顧}
    \\ 針對該領域現存的研究進行介紹與整理,包括人體量測、模擬與分析相關研究,以及個人化模型的必要性,
    並針對肌肉參數評估等文獻,進行回顧與討論。
    \item \textbf{第三章:研究方法}
    \\ 介紹本研究所提出之參數評估方法,包含敏感度分析、參數評估與最佳化,以及模型驗證方法,
    除此之外也介紹了相關背景知識,例如肌肉模型、動力學模擬、肌肉計算控制模擬等資訊,藉由這整套方法來建立個人化模型。
    \item \textbf{第四章:上肢特定肌肉之參數評估與最佳化}
    \\ 針對第三章所提出之方法,使用普及的上肢肌肉骨骼模型作為範例,進行相關的前置作業與研究方法的套用說明。
    \item \textbf{第五章:參數評估模擬案例與成果探討}
    \\ 綜合第三章的研究方法與第四章的骨骼模型,呈現數個模擬案例來顯現該方法的有效性與必要性,
    同時驗證所評估參數之正確性,並對參數的評估結果進行探討。
    \item \textbf{第六章:結論與未來工作}
    \\ 總結本研究之成果與貢獻,並給予適當的建議,作為未來本研究之延伸方向。
\end{itemize}

\clearpage