\chapter{相機擺放位置對於人體姿態估計誤差之影響}
\fontsize{12pt}{18pt}\selectfont

% ------------------------- 5.0 ------------------------- %
% 概述;本章安排
本章節基於章節~\ref{ch4_sec_cameraset} 的實驗結果,進一步就 TotalCapture Dataset 架設的相機位置,探討相機擺放位置對於人體姿態估計誤差之影響,並進行討論。

% ------------------------- 5.1 ------------------------- %
\section{實驗設定}
由於目前學者 Zhe Zhang 等人取用四台相機~\cite{Zhang_2020_CVPR},因此若欲減少相機使用數量勢必會往三台相機及兩台相機的配置進行考量,因此以下將進一步比較三台相機及兩台相機的估計結果,並進行討論。

\subsubsection*{方法}
將三台相機的估計結果依照兩台相機的配對進行平均,再與兩台相機的結果相減,以此方式觀察多一台相機的影響。
計算方式如算式~\ref{ch3_3camaveMin2cam}。
\begin{equation}
   \label{ch3_3camaveMin2cam}
   \text{差值} = \text{ave}(\text{MPJPE}_{3 \text{ cam}})-\text{MPJPE}_{2 \text{ cam}}
\end{equation}

\section{結果與討論}
將本文中每一相機數量的最佳表現及最差表現一一列出,如表~\ref{ch3_best_worst_camset},可以發現相機 1 皆有出現在最佳表現的結果中,而相機 2 皆有出現在最差表現的結果中,因此推論相機 1 的位置較為適當。
\begin{table}[!ht]
   \caption[相機組合的最差及最佳表現]{相機組合的最差及最佳表現}
   \centering
   \label{ch3_best_worst_camset}
   \setlength{\tabcolsep}{3pt}
   \renewcommand\arraystretch{1.5}
   \begin{tabular}{c|c|c|c|c}
       & 1 cam & 2 cam & 3 cam & 4 cam \\ 
      \midrule[2pt]
      最差表現 & 2 & 25 & 257 & 2567 \\
      最佳表現 & 1 & 18 & 137 & 1357 \\
   \end{tabular}
\end{table}

另外,以下表~\ref{ch3_ave_3cam_vs_2cam} 交互相機 18 為例,即將 128、138、148、158、168、178 六組相機配對的估計結果進行平均,得 MPJPE = 34.5704,再與直接用兩台相機進行融合估計得到的 MPJPE = 46.9798 相減,得 12.4094。由下表~\ref{ch3_ave_3cam_vs_2cam} 可知,相機 1 及相機 8 的配對再加上第三台相機較不會有顯著的改善,且前三種配對 (即配對 18、配對 14、配對 48)的改善皆都落在約 10~15 mm 左右,所以可以推斷,相機 1、4、8 三個位置,隨意取其中兩者可能是較適合擺放相機的位置。

\begin{table}[!ht]
   \caption[兩台相機與三台相機平均之比較]{兩台相機與三台相機平均之比較}
   \centering
   \label{ch3_ave_3cam_vs_2cam}
   \setlength{\tabcolsep}{3pt}
   \renewcommand\arraystretch{1.5}
   \begin{tabular}{c|c|c|c}
      交互相機 & $ave(MPJPE_{3 cam})$ & $MPJPE_{2 cam}$ & 差 \\
      \midrule[2pt]
      18 & 34.5704 & 46.9798 & 12.4094 \\
      14 & 33.8308 & 47.4927 & 13.6619 \\
      48 & 35.3844 & 52.3318 & 16.9474 \\
      17 & 34.9321 & 58.7619 & 23.8298 \\ 
      58 & 38.4382 & 64.6374 & 26.1993 \\
      68 & 35.3408 & 63.7904 & 28.4496 \\
      46 & 37.9318 & 66.8800 & 28.9481 \\  
      47 & 38.0495 & 67.4949 & 29.4454 \\ 
      16 & 35.3793 & 66.8653 & 31.4860 \\ 
      15 & 39.8348 & 76.7795 & 36.9447 \\
      78 & 39.5911 & 77.1948 & 37.6037 \\
   \end{tabular}
\end{table}

最後,若希望 $MPJPE_{2 cam}$ 低於 80 mm,則從兩台相機的估計結果(表~\ref{ch3_cameraset_2cam})挑選出 $MPJPE_{2 cam}$ 低於 80 mm 的結果,並計算每一相機的出現次數,則可得出如下表~\ref{ch3_cam_occurrence} 的結果,觀察下表也可發現相機 1、8 的出現次數最高,其次為相機 4 ,因此也可推斷相機 1、4、8 三個位置可能是較適合擺放相機的位置。

\begin{table}[!ht]
   \caption[兩台相機估計結果之每台相機的出現次數]{兩台相機估計結果之每台相機的出現次數}
   \centering
   \label{ch3_cam_occurrence}
   \setlength{\tabcolsep}{3pt}
   \renewcommand\arraystretch{1.5}
   \begin{tabular}{c|c}
      相機 & 出現次數 \\
      \midrule[2pt]
      1 & 5 \\
      2 & 0 \\
      3 & 0 \\
      4 & 4 \\
      5 & 2 \\
      6 & 3 \\
      7 & 3 \\
      8 & 5 \\
   \end{tabular}
\end{table}

\section{小結}
% 總結
綜上所述,本章節透過 TotalCapture Dataset 架設的相機位置,探討相機擺放位置對於人體姿態估計誤差之影響,並進行討論。由實驗結果可知,相機 1、4、8 三個位置可能是較適合擺放相機的位置,且相機 1 的位置較為適當,且由兩台相機配對時,相機 1 與相機 8 的表現最佳,可以推論,在架設實驗場地時,在僅使用兩台相機的情況下可以優先分別將相機架設於面向受試者的位置。

\clearpage