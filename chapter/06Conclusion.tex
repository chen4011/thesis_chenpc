\chapter{結論與未來工作}
\fontsize{12pt}{18pt}\selectfont

% ------------------------- 6.0 ------------------------- %
在本論文中,首先第一章介紹了關於人體動作模擬與分析的研究背景,以及其必要性與應用場合,衍生出本論文之動機與目的;
接續第二章介紹相關的背景文獻,也針對相似的文獻進行回顧與探討;在第三章的部分,提出了本論文之核心研究方法,
透過這套研究方法來對肌肉骨骼模型進行肌肉參數估計,目的是為了建立個人化模型,過程包括預測任務、敏感度分析、
多運動軌跡預測最佳化,以及模型驗證;第四章與第五章將該研究方法套用在一個普及的上肢肌肉骨骼模型中,
透過數個多軌跡案例來展示該方法的有效性與正確性,並以單軌跡案例來證實解決肌肉參數具不識別性的重要性。
最後本章節將統整論文,依序列出本研究成果與貢獻,以及提出數個未來工作,作為本研究可延伸之方向。

% ------------------------- 6.1 ------------------------- %
\section{研究成果與貢獻}
下方將條列出數個關於本研究之成果與貢獻,讓讀者除了回顧本論文之核心方法外,也能清楚瞭解該論文之成果與貢獻處。

\begin{itemize}
    \item \textbf{透過現有開源軟體進行人體動作模擬與分析研究}
    \\ 本研究使用現有的 OpenSim 開源生物力學模擬軟體,將生物力學領域的計算全權交由其處理,
    OpenSim 在該領域已被學者廣泛使用,其結果相對具有可信度,再結合 MATLAB 商業數學軟體其基本數學運算、最佳化演算法,
    以及平行運算等功能,藉由呼叫 OpenSim API 來完成整個研究的模擬與分析,達到肌肉參數估計、建立個人化模型的最終目的。
    \clearpage
    \item \textbf{提出最佳化方法同時評估多條肌肉之多參數組合}
    \\ 本研究核心方法是透過預測任務來完成肌肉參數的估計,藉由目標運動軌跡與預測軌跡的不斷比較,
    來尋找出能產生與目標軌跡最相似的預測模型,作為最終的個人化模型,與此同時也完成了肌肉參數的估計。
    在這之中提出了多運動軌跡預測最佳化方法,藉由結合多預測任務與最佳化演算法,來完成研究。
    \item \textbf{結合敏感度分析來尋找最佳化與驗證所需之特定動作}
    \\ 指派無意義之任務進行參數評估是徒勞無功的,本研究透過敏感度分析來量化任務與肌肉間之關聯,
    供最佳化與驗證模型階段的任務挑選參考依據,除了能確認任務是否符合作為評估對象外,亦提高驗證模型的效果。
    \item \textbf{探討肌肉參數的不可識別性問題}
    \\ 肌肉參數的改變可由其餘參數的變動來抗衡 \cite{bujalski2018monte},該點說明了肌肉參數具不可識別性問題,
    本研究於單軌跡案例中展示了該性質,縱使預測任務的誤差非常小,仍有可能產生一組與標準答案偏離許多的解,
    因此單一任務的預測是不可行的,故透過多預測任務來解決該問題。
\end{itemize}

% ------------------------- 6.2 ------------------------- %
\section{未來工作}
下方將列舉幾個未來工作,可作為本研究之延伸方向:

\begin{itemize}
    \item \textbf{提高欲評估之肌肉參數數量}
    \\ 本研究最終能估計兩條肌肉,共含有六個肌肉參數是同時被評估的,該研究成果雖踏出了一步,
    但對於建立全身的個人化模型仍有段距離,本文作者認為可先朝同時評估上臂所有肌肉作為目標,
    也就是先以 arm26 所提供之六條肌肉同時評估作為未來短期目標。
    \item \textbf{提高人體模擬動作自由度與複雜度}
    \\ 在動作任務的選定上,本研究侷限於肘關節的單自由度轉動,若能提高任務的複雜度,雖可能提高執行複雜度,
    但或許能更有效的估計肌肉參數。
    \item \textbf{預測任務數量與評估結果間的關係}
    \\ 在多運動軌跡預測最佳化中,多軌跡案例以同時執行兩個預測任務作為範例,在未來工作中,
    建議學者能往預測任務的數量進行探討,預測任務數量的增加會使每次的迭代時間增長,
    但若能使演算法提早找到最佳參數,則會是更好的方法,因此預測任務的數量是值得探討的。
    \item \textbf{結合全域敏感度分析來尋找最佳化與驗證所需之最合適動作}
    \\ 本研究的敏感度分析只單就每個任務對每條肌肉的影響,若能執行全域敏感度分析,
    則可提供更明確、合適的任務,作為參數評估的使用,藉此提高整個方法的速度與正確性。
    \item \textbf{透過臨床實驗來檢視該方法的成效}
    \\ 由於現今的量測技術仍有許多的量測誤差存在,這些誤差就足以因為參數的不識別性,產生許多不同的個人化模型,
    因此待量測技術更加完善後,應進行臨床實驗,來檢視該方法的成效。
\end{itemize}

\clearpage