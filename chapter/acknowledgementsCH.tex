\begin{acknowledgementsCH}
    \fontsize{12pt}{18pt}\selectfont
    整本論文僅剩誌謝需要撰寫是多麼奢侈的一件事。首先感謝百忙抽空來擔任考試委員的徐瑋勵老師與張秉純老師,
    針對研究與論文提出寶貴建議,使本論文更加完整。謝謝在碩士班期間指導我的詹魁元老師,
    老師總是希望我們能主動去尋找自己的研究方向,雖然有時會有些天馬行空的想法,
    從碩一的機械手臂、步態分析、IMU 動作量測、訊號處理,研究主題跟計劃助理的更換頻率一樣高,直到碩二才逐漸穩定,
    但正是在這飄搖不定的研究主題中探索,也才能真正體會到研究的五味雜陳,除了感謝老師的指導外,也謝謝老師適時的給予信心,
    還有能加入 SOLab 的機會,很開心也很榮幸地順利完成了碩士學位。

    長幼有序,首先感謝實驗室的大家長彥智學長,謝謝彥智在研究上的各種協助,像是回答一些在最佳化上的蠢問題,
    感謝一同共患難兩年的柏賢,不管是在運動領域上的常識補充,還是這兩年來的生活瑣碎,都給予了很大的協助,
    敬祝兩位博班學長在未來研究上一帆風順。謝謝詔東在修課與程式上的指導、昱凡在研究上的莫大大大幫助 (尤其是畢業後的學術騷擾)、
    冠成的好奇心使我在研究上的啟發,還有亭宜在研究上的鼓勵,撇除學術上的感謝外,從喝酒、打牌、開趴、出遊,再到電影、火鍋馬拉松,
    學長姊們帶來更多的是生活上的樂趣與動力來源。謝謝同屆一起奮鬥的夥伴們,啟瑞在日常中的閒聊與協助、
    怡萱在千奇百怪問題上的解惑、若瑄在修課期間的各種組隊、重叡都不講話 (但後期真.共患難了數次),以及冠賢在程式上的協助與日常討論,    
    雖每個人的研究領域都隔了一座山,或許不能相互給予實質幫助,但在心靈上絕對是大拇指的,同時也讓我見識到了何謂強者。
    最後也謝謝琮祐、鐘毅、敬亭、問蕖、珮甄、怡葶,協助處理實驗室大小事,祝順利畢業。

    感謝在碩士兩年相互扶持的朋友們,像是同為從成大上來臺大的啟玄、志霖、敬桓和立渝,歡樂的吃飯時光、
    羽球團和爬山都給予生活滿滿動力,雖然說了兩年的基隆行還沒成,另外也還有永和酒友會的名宇和冠丞。
    謝謝高中好麻吉兼吃飯好友振原,在剛來臺大時的協助與後期的職涯關心,謝謝嘉宏、餘之、晴立,
    以及其他仍保持聯繫的朋友們,分享日常瑣碎與相互關心都是菸酒生的生活調劑。此外研究生加入滑板社看似荒謬,
    但卻是我在研究最苦悶時的精神糧食,在此特別感謝佐庭、銘志和品竹。最要感謝的是我的家人,除了實質上的經濟援助外,
    還有心靈上的鼓舞,不僅僅是碩士兩年,而是從出生那刻到現在,沒有您們的栽培,我就不會有現在如此的成就。

    最後,感謝我自己願意相信自己、堅持到底,小時候對臺大的印象是遙不可及的,如今卻在此順利地完成了求學里程,
    期許自己在未來能找到屬於自己合適的位置並發光發熱,不管是在職場、社會,還是自己的人生當中。

    \begin{flushright}
        林易玄 謹誌於\\
        國立臺灣大學 機械工程學系\\
        中華民國一百一十二年七月
    \end{flushright}
 
\end{acknowledgementsCH}