\chapter{誌謝}
    \fontsize{12pt}{18pt}\selectfont
    經歷了迷惘的一年級和焦慮又緊張的二年級,終於來到了換我寫誌謝的此時此刻,在這兩年間,感覺自己成長了很多,
    不管是專業方面的知識,還是人際關係的處理,都讓我學到了很多,
    最首先就是要謝謝詹魁元老師,謝謝老師在這兩年間的指導與教導,每次去找他討論問題時,老師總能給我許多正向的肯定和回復,
    還記得在這半年內發生了許多事,一開始是發現已經開源的程式不如預期,可能需要重新尋找工具,是老師告訴我還有希望,
    先試試看有沒有辦法改掉出問題的部分,最壞打算就是自己寫一組工具,時間還來得及;
    還有一次是在二年級下學期的後半學期,這種緊要關頭的時刻,我的研究進度卡了兩週,嘗試多種方法,也嘗試尋找文獻,但都無法解決,
    老師聽完我的困難後,他並沒有任何指責,反而是告訴我,這個問題他可能也無法直接就幫我解決,但是我們可以找相同領域的人一起討論,
    我知道此時此刻真的很無助,但是我們可以一起找解決方法,這樣的回應讓我感到很溫暖,也讓我知道,我們不是孤單的在這條路上走,
    謝謝老師在研究過程中給我許多信心與肯定,讓我覺得自己也能辦到,讓我能夠堅持到最後。

    再來要謝謝實驗室的學長、同學、學弟妹,謝謝 Paul 學長在最後寫論文的時刻跟我討論整體論文的架構,讓我能夠更清楚的知道論文的方向;
    謝謝張問蕖總是願意跟我討論我的問題,提出他的想法和可能的解決方法,讓我能夠更快的找到問題的解決方法;
    謝謝冠賢從念研究所的動機、選擇研究的方向、論文的動機,甚至是人生的方向、規劃,都願意跟我分享,讓我能夠更清楚的知道自己想要的是什麼;
    謝謝怡葶跟我討論研究中的細節和問題,謝謝柯琮祐陪我在永齡待到半夜,再被鐵門關在永齡裡面,謝謝謝鐘毅和蕭敬亭在我研究遇到困難時,願意跟我討論我的問題,
    謝謝蘇瑄、定群、邑安、京睿、家安,雖然我們相處的時間不多,但感謝你們處理實驗室的大小事,讓我們能夠有一個舒適的研究環境。
    還有謝謝黃廷睿,即使你也要趕自己的畢業論文畢業口試,還是願意花時間安撫我焦慮的心情,也花很多很多時間幫忙我釐清整個程式的邏輯,
    幫助我理解整支程式的運作和整支程式的重點,甚至花許多時間協助我蒐集研究數據,出借實驗場地,
    謝謝你在我最需要幫助的時候,即使自己快要分身乏術,還是會不厭其煩的幫助我所有的事情。
    最後也要謝謝勇敢堅持到現在的自己,過程中雖然有許多困難、挫折、焦慮,但還是努力的完成了這段學業。
    最後也要謝謝老媽,為了不造成我的壓力,每次都小心翼翼的關心著我、擔心著我,
    也要謝謝老爸,定時提供金援,從不間斷,讓我不用為了生活的問題而分心,讓我能夠專心的完成學業。
    
    這一刻的情緒十分複雜,有終於把學業告一個段落的喜悅,也有接下來就要步入職場的無措,
    雖然不知道未來會遇到什麼樣的困難,但是我相信,這兩年來的經驗,讓我有能力去面對未來的挑戰,
    也相信未來的路,會有更多的人和事,讓我成長,讓我更加堅強,謝謝這兩年來所有幫助過我的人,讓我能夠順利完成這段學業。

    \begin{flushright}
        陳珮甄 謹誌於\\
        國立臺灣大學 機械工程學系\\
        中華民國一百一十三年七月
    \end{flushright}
 
% \end{acknowledgementsCH}